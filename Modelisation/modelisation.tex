\documentclass[11pt,article]{article}
\usepackage[utf8]{inputenc}
\usepackage[T1]{fontenc} % caractères accentués en entrée, dans emacs
\usepackage[french]{babel}
\FrenchFootnotes
\selectlanguage{french}
\usepackage{a4wide} % possibilité d'utiliser toute la page a4
% selon GUT#33, avril 2007, page 13, empagement
% largeur des textes (ou justification) = 15cm
% hauteur du rectangle d'empagement = 23cm
% blanc de couture = 2/5 (21-15) = 2.4 = inner = right
% blanc de grand fond = 3/5 (21-15) = outer = left
% blanc de tête = 2/5 (29,7-23) = top
% blanc de pied = 3/5 (29,7-23) = bottom
%\usepackage[a4paper,twoside=true,right=2.4cm,left=3.6cm,top=2.68cm,bottom=4.02cm]{geometry} 
% selon CFSE 2006
% - largeur des textes (ou justification) : 16cm (2cm de marge, et 1cm
%   de reliure) ;
% - hauteur des textes, y compris les notes : 23cm (2,5cm de marge
%   haute et 2cm de marge basse) ; 1ère page de : 36pts
%   d'espacement avant le titre ;
\oddsidemargin   -4mm           % 3cm a gauche des impaires
\evensidemargin   4mm           % 2cm a gauche des paires
\topmargin       -18mm          % 2.5cm en haut
\headheight       13mm          % taille de l'entete (lignes)
\headsep          24pt          % espace entre entete et texte
\footskip         30pt          % espace entre pied de page et texte
\textheight      230mm          % longeur du texte
\textwidth       160mm          % largeur du texte
\parskip 1pt                    % pas de sauts entre paragraphes
%\parindent 0pt                  % largeur de l'indentation
\usepackage{graphicx} % figure postcript avec latex,
		      % figure png avec pdflatex, au lieu d'utiliser epsfig
\usepackage[usenames,dvipsnames,table]{xcolor}
\usepackage{paralist}
\usepackage{ifthen}
\usepackage{amssymb}
\usepackage{amsfonts}
\usepackage{amsmath}
\usepackage{eurosym}
\usepackage{textcomp}
\usepackage{listings}
\lstset{language=Java,numbers=left,numberstyle=\tiny,stepnumber=4,numbersep=5pt,xleftmargin=5pt}

\usepackage{alltt}
\usepackage{longtable}

% adjust word spacing less strictly
% as result, some spaces between words may be a bit too large,
% but long words will be placed properly.
\sloppy

\newcommand{\cmt}[1]{\texttt{<}\textbf{--~#1~--}\texttt{>}}

\usepackage{lineno}
\usepackage{xspace}

\setlength{\marginparwidth}{1cm}
\setlength{\marginparsep}{10pt}
\reversemarginpar
\newcounter{usecasehaute}
\newcommand{\haute}{Haute}
\newcommand{\moyenne}{Moyenne}
\newcommand{\basse}{basse}
\newcommand{\usecase}[4]{\item \marginpar{\vspace{5pt}\ifthenelse{\equal{#1}{Haute}}{\centering\textsc{#1}\stepcounter{usecasehaute}\newline n$^{\circ}$ \theusecasehaute}{\ifthenelse{\equal{#1}{Moyenne}}{#1}{\small #1}}} #2 \begin{itemize}\item précondition~: #3 \item postcondition~: #4\end{itemize}}
\newcommand{\priorityusecase}[2]{\item \marginpar{\vspace{5pt}\ifthenelse{\equal{#1}{Haute}}{\centering\textsc{#1}\stepcounter{usecasehaute}\newline n$^{\circ}$ \theusecasehaute}{\ifthenelse{\equal{#1}{Moyenne}}{#1}{\small #1}}} #2}
\newcommand{\casusecase}[4]{\usecase{#1}{#2}{#3}{#4}}

\newcommand{\nullvalue}{\textsf{null}\xspace}
\newcommand{\emptyvalue}{\ensuremath\mathrm{vide}}
\newcommand{\invariant}{\ensuremath\mathrm{invariant}}

\begin{document}
\title{Projet CSC4102: Gestion de bureaux en situation de bilocalisation}
\author{Nom Prénom Étudiant1 et Nom Prénom Étudiants2}
\date{Année 2020--2021~---~\today}
\maketitle

\newpage

\tableofcontents

\newpage

\section{Spécification}

\subsection{Diagrammes de cas d'utilisation}

{\color{red}\textbf{Le diagramme suivant est à compléter.}}

\begin{figure}[h!]
\begin{center}
\includegraphics[scale=0.5]{DiagrammesDeCasDUtilisation/bebiloc_uml_diag_cas_utilisation}
\caption{Diagramme de cas d'utilisation}
\end{center}
\label{umlet_diag_cas_utilisation}
\end{figure}

\newpage

\subsection{Priorités, préconditions et postconditions des cas d'utilisation}

Les priorités des cas d'utilisation pour le sprint~1 sont choisies
avec les règles de bon sens suivantes:
\begin{compactitem}
\item pour retirer une entité du système, elle doit y être. La
priorité de l'ajout est donc supérieure ou égale à la priorité du
retrait;
\item pour lister les entités d'un type donné, elles doivent y être. La
priorité de l'ajout est donc supérieure ou égale à la priorité du
listage;
\item il est \textit{a priori} possible, c.-à-d. sans raison
contraire, de démontrer la mise en œuvre d'un sous-ensemble des
fonctionnalités du système, et plus particulièrement la prise en
compte des principales règles de gestion, sans les retraits ou les
listages.
\item la possibilité de lister aide au déverminage de l'application
pendant les activités d'exécution des tests de validation.
\end{compactitem}
Par conséquent, les cas d'utilisation d'ajout sont \textit{a priori}
de priorité <<~haute~>>, ceux de listage de priorité
<<~moyenne~>>, et ceux de retrait de priorité <<~basse~>>.

\bigskip

Dans la suite, nous donnons les préconditions et postconditions pour
les cas d'utilisation de priorité <<~\haute~>>. Pour les autres, nous
indiquons uniquement leur niveau de priorité.

\bigskip

{\color{red}\textbf{La liste de préconditions et postconditions
    suivante est à compléter.}}

\bigskip

Nous avons préféré mettre deux cas d'utilisation pour ajouter un
permanent ou un non-permanent, plutôt qu'un unique cas d'utilisation
pour ajouter un employé. L'avantage des deux cas d'utilisation est
d'éviter des données en entrée non utilisées dans certaines
utilisations: un employé permanent ne possède pas de date de fin de
contrat. Si l'on décide de mettre un seul cas d'utilisation, alors il
faut vérifier (1)~qu'un permanent n'a pas de date de fin de contrat,
et (2)~qu'un non-permanent possède une date de fin de contrat. C'est
une préférence: cela n'impacte pas les autres cas d'utilisation.

\bigskip

\begin{compactitem}
\usecase{\haute}{Ajouter un employé permanent}
        %% précondition
        {\newline
          $\land$ identifiant de l'employé bien formé (non \nullvalue
          et non vide)
          \newline
          $\land$ nom bien formé (non \nullvalue et non vide)
          \newline
          $\land$ prénom bien formé (non \nullvalue et non vide)
          \newline
          $\land$ date d'embauche non \nullvalue
          \newline
          $\land$ fonction du permanent bien formée (non \nullvalue
          et non vide)
          \newline
          $\land$ fonction du permanent $\in \{$ direction
          département, direction adjointe département, assistance
          gestion, enseignement recherche $\}$
           \newline
          $\land$ employé avec cet identifiant inexistant}
        %% postcondition
        {employé permanent avec cet identifiant existant}

\smallskip

\usecase{\haute}{Ajouter un employé non-permanent}
        %% précondition
        {\newline
          $\land$ identifiant de l'employé bien formé (non \nullvalue
          et non vide)
          \newline
          $\land$ nom bien formé (non \nullvalue et non vide)
          \newline
          $\land$ prénom bien formé (non \nullvalue et non vide)
          \newline
          $\land$ date d'embauche non \nullvalue
          \newline
          $\land$ date de fin de contrat non \nullvalue
          \newline
          $\land$ date de fin de contrat $\geq$ date d'embauche
          \newline
          $\land$ fonction du non-permanent bien formée (non \nullvalue
          et non vide)
          \newline
          $\land$ fonction du non-permanent $\in \{$ doctorat,
          post-doctorat, ingénierie recherche, stage $\}$
          \newline
          $\land$ employé avec cet identifiant inexistant}
        %% postcondition
        {employé permanent avec cet identifiant existant}
\end{compactitem}

\newpage

\section{Préparation des tests de validation}

\subsection{Tables de décision des tests de validation}

La fiche programme du module CSC4102 ne permettant pas de développer
des tests de validation couvrant l'ensemble des cas d'utilisation de
l'application, les cas d'utilisation choisis sont ceux de priorité
\textsc{Haute}.

\bigskip

{\color{red}\textbf{La section est à compléter avec les tables de
    décision d'autres cas d'utilisation.}}

\bigskip

\begin{table}[htbp!]
\begin{tabular}{|p{0.6\linewidth}|c|c|c|c|c|c|c|c|c|}
\hline
Numéro de test
&1&2&3&4&5&6&7&8\\
\hline
\hline
Identifiant de l'employé bien formé (non \nullvalue et non vide)
&F&T&T&T&T&T&T&T\\
\hline
Nom bien formé (non \nullvalue et non vide)
& &F&T&T&T&T&T&T\\
\hline
Prénom bien formé (non \nullvalue et non vide)
& & &F&T&T&T&T&T\\
\hline
Date d'embauche $\neq$ \nullvalue
& & & &F&T&T&T&T\\
\hline
Fonction du permanent bien formée (non \nullvalue et non vide)
& & & & &F&T&T&T\\
\hline
fonction du permanent $\in \{$ direction département, direction adjointe département, assistance gestion, enseignement recherche $\}$
& & & & & &F&T&T\\
\hline
Employé avec cet identifiant non existant
& & & & & & &F&T\\
\hline
\hline
Création acceptée
&F&F&F&F&F&F&F&T\\
\hline
\hline
Nombre de jeux de test 
&2&2&2&1&2&5&1&4\\
\hline
\end{tabular}
\caption{Cas d'utilisation <<~ajouter un employé permanent~>>. Le
  test~6 possède 5 jeux de test pour les 4 fonctions de non-permanents
  et pour une fonction inconnue. Le test~8 possède 4 jeux de test pour
  les 4 fonctions de permanents.}
\end{table}

\begin{table}[htbp!]
\begin{tabular}{|p{0.6\linewidth}|c|c|c|c|c|c|c|c|c|c|}
\hline
Numéro de test
&1&2&3&4&5&6&7&8&9&10\\
\hline
\hline
Identifiant de l'employé bien formé (non \nullvalue et non vide)
&F&T&T&T&T&T&T&T&T&T\\
\hline
Nom bien formé (non \nullvalue et non vide)
& &F&T&T&T&T&T&T&T&T\\
\hline
Prénom bien formé (non \nullvalue et non vide)
& & &F&T&T&T&T&T&T&T\\
\hline
Date d'embauche $\neq$ \nullvalue
& & & &F&T&T&T&T&T&T\\
\hline
Date de fin de contrat $\neq$ \nullvalue
& & & & &F&T&T&T&T&T\\
\hline
Date de fin de contrat $\geq$ date d'embauche
& & & & & &F&T&T&T&T\\
\hline
Fonction du non-permanent bien formée (non \nullvalue et non vide)
& & & & & & &F&T&T&T\\
\hline
Fonction du non-permanent $\in \{$ doctorat, post-doctorat, ingénierie recherche, stage $\}$
& & & & & & & &F&T&T\\
\hline
Employé avec cet identifiant non existant
& & & & & & & & &F&T\\
\hline
\hline
Création acceptée
&F&F&F&F&F&F&F&F&F&T\\
\hline
\hline
Nombre de jeux de test 
&2&2&2&1&1&1&2&5&1&4\\
\hline
\end{tabular}
\caption{Cas d'utilisation <<~ajouter un employé non-permanent~>>. Le
  test~8 possède 5 jeux de test pour les 4 fonctions de permanents et
  pour une fonction inconnue. Le test~10 possède 4 jeux de test pour
  les 4 fonctions de non-permanents.}
\end{table}

\newpage

\section{Conception}

\subsection{Liste des classes}

{\color{red}\textbf{La liste des classes suivante est à compléter.}}

\bigskip

À la suite d'un parcours des diagrammes de cas d'utilisation et d'une
relecture de l'étude de cas, voici la liste de classes avec quelques
attributs:
\begin{compactitem}
\item \textsf{BeBiloc} (la façade),
\item \textsf{Employé}~---~identifiant, nom, prénom, dateEmbauche, dateFinContrat, fonction,
\item \textsf{Fonction} (énumération)~---~nom, typeFonction, avec 8 énumérateurs,
\item \textsf{TypeFonction} (énumération)~---~2 énumérateurs
\item ...
\end{compactitem}

\bigskip

Une variante serait une généralisation spécialisation au lieu d'une
énumération autour de la classe \textsf{Employé}: \textsf{Employé} en
classe parente, et \textsf{DIRECTION\_DÉPARTEMENT}, etc. en classes
enfants. Nous avons préféré une classe avec une énumération car un
employé peut changer de fonction (ce cas d'utilisation est plus
fréquent qu'on ne le croit) et beaucoup de traitements dépendent du
type de l'employé (l'utilisation d'une énumération est quelque peu
plus aisée).

\newpage

\subsection{Diagramme de classes}

{\color{red}\textbf{Le diagramme de classes suivant est à compléter.}}

\begin{figure}[h!]
\begin{center}
\includegraphics[scale=0.6]{DiagrammesDeClasses/bebiloc_uml_diag_classes}
\caption{Diagramme de classes}
\end{center}
\label{umlet_diag_classes}
\end{figure}

\newpage

\subsection{Diagrammes de séquence}

{\color{red}\textbf{La section est à compléter avec les diagrammes de séquence de vos cas d'utilisation les plus importants, c'est-à-dire avec ceux de priorité haute.}}

\bigskip

\noindent
Voici la description textuelle du cas d'utilisation <<~ajouter un employé permanent~>>:
\begin{compactitem}
\item arguments en entrée: identifiant de l'employé, nom et prénom de
  l'employé, date d'embauche, date de fin de contrat, fonction;
\item rappel de la précondition: identifiant de l'employé bien formé
  (non \nullvalue et non vide) $\land$ nom bien formé (non \nullvalue
  et non vide) $\land$ prénom bien formé (non \nullvalue et non vide)
  $\land$ date d'embauche non \nullvalue $\land$ fonction du permanent
  bien formée (non \nullvalue et non vide) $\land$ fonction du
  permanent $\in \{$ direction département, direction adjointe
  département, assistance gestion, enseignement recherche $\}$ $\land$
  employé avec cet identifiant inexistant
\item algorithme:
\begin{compactenum}
\item vérifier les arguments
\item chercher un employé avec cet identifiant
\item vérifier que l'employé est inexistant
\item instancier l'employé
\item ajouter l'employé dans la collection des employés
\end{compactenum}
\end{compactitem}

\begin{figure}[ht!]
\begin{center}
\includegraphics[scale=0.5]{DiagrammesDeSequence/bebiloc_uml_diag_sequence_ajouter_permanent}
\caption{Diagramme de séquence du cas d'utilisation <<~ajouter un
  employé permanent~>>}
\end{center}
\label{umlet_diag_sequence_ajouter_non_permanent}
\end{figure}

\newpage~\newpage

\noindent
Voici la description textuelle du cas d'utilisation <<~ajouter un employé non-permanent~>>:
\begin{compactitem}
\item arguments en entrée: identifiant de l'employé, nom et prénom de
  l'employé, date d'embauche, date de fin de contrat, fonction;
\item rappel de la précondition: identifiant de l'employé bien formé
  (non \nullvalue et non vide) $\land$ nom bien formé (non \nullvalue
  et non vide) $\land$ prénom bien formé (non \nullvalue et non vide)
  $\land$ date d'embauche non \nullvalue $\land$ date de fin de
  contrat non \nullvalue $\land$ date de fin de contrat $\geq$ date
  d'embauche $\land$ fonction du non-permanent bien formée (non
  \nullvalue et non vide) $\land$ fonction du permanent $\in
  \{$doctorat, post-doctorat, ingénierie recherche, stage$\}$ $\land$
  employé avec cet identifiant inexistant
\item algorithme:
\begin{compactenum}
\item vérifier les arguments
\item chercher un employé avec cet identifiant
\item vérifier que l'employé est inexistant
\item instancier l'employé
\item ajouter l'employé dans la collection des employés (non modélisé ici)
\end{compactenum}
\end{compactitem}

\newpage

\begin{figure}[ht!]
\begin{center}
\includegraphics[scale=0.45]{DiagrammesDeSequence/bebiloc_uml_diag_sequence_ajouter_non_permanent}
\caption{Diagramme de séquence du cas d'utilisation <<~ajouter un
  employé non-permanent~>>}
\end{center}
\label{umlet_diag_sequence_ajouter_non_permanent}
\end{figure}

\newpage

\section{Fiche des classes}

{\color{red}\textbf{La section est à compléter avec les fiches de vos
    classes les plus importantes. La première fiche, celle de la
    façade, est aussi à compléter. La seconde fiche, celle de la
    classe \textsf{Employé}, est aussi à compléter.}}

\subsection{Classe \textsf{BeBiloc}}

\begin{center}
\begin{longtable}{|p{15cm}|} 
\hline
\multicolumn{1}{|c|}{{\Large \textsf{BeBiloc}}} \\
\hline
%\cmt{attributs}\\
\cmt{attributs <<~association~>>}\\
$-$ employes : collection de Employe \\
\hline
\cmt{constructeur} \\
$+$ BeBiloc()\\
%$+$ destructeur()\\
$+$ invariant() : booléen\\
\cmt{operations <<~cas d'utilisation~>>} \\
$+$ ajouterPermanent(String id, String nom, String prenom, LocalDate dateEmbauche, String fonction) \\
$+$ ajouterNonPermanent(String id, String nom, String prenom,
LocalDate dateEmbauche, LocalDate dateFinContrat, String fonction) \\
$+$ listerEmployes() : collection de String \\
%\cmt{opérations de recherche} \\
\hline  
\end{longtable}%)
\end{center}

\subsection{Classe \textsf{Employé}}

\begin{center}
\begin{longtable}{|p{15cm}|} 
\hline
\multicolumn{1}{|c|}{{\Large \textsf{Employé}}} \\
\hline
%\cmt{attributs}\\
$-$ identifiant : String \\
$-$ nom : String \\
$-$ prénom : String \\
$-$ dateEmbauche : Date \\
$-$ dateFinContrat : Date \\
$-$ fonction : Fonction \\
\cmt{attributs <<~association~>>}\\
\hline
\cmt{constructeur} \\
$+$ Employé(String id, String nom, String prénom, LocalDate
dateEmbauche, Fonction fonction)\\
$+$ Employé(String id, String nom, String prénom, LocalDate
dateEmbauche, LocalDate dateFinContratFonction fonction)\\
%$+$ destructeur()\\
$+$ invariant() : booléen\\
\hline  
\end{longtable}%)
\end{center}

\newpage

\section{Diagrammes de machine à états et invariants}

{\color{red}\textbf{La section est à compléter avec les diagrammes de
    machine à états et les invariants de vos classes les plus
    importantes.}}

\subsection{Classe \textsf{Employé}}

\textbf{\color{red}L'invariant est à compléter}

\begin{figure}[ht!]
\begin{center}
\includegraphics[scale=0.35]{DiagrammesDeMachineAEtats/bebiloc_uml_diag_machine_a_etats_employe}
\caption{Diagramme de machine à états de la classe \texttt{Employé}}
\end{center}
\label{umlet_diag_machine_a_etats_employe}
\end{figure}

Pour garder l'invariant quelque peu local à la classe, c'est-à-dire
sans trop d'appels sur d'autres objets métier, les conditions
suivantes ne sont pas intégrées à l'invariant:
\begin{compactitem}
\item le nombre de places de passage est 1,
\item le nombre de places sur un même site est 1.
\end{compactitem}

\bigskip

\newcommand{\identifiant}{\ensuremath\mathrm{identifiant}}
\newcommand{\nom}{\ensuremath\mathrm{nom}}
\newcommand{\prenom}{\ensuremath\mathrm{prenom}}
\newcommand{\dateEmbauche}{\ensuremath\mathrm{dateEmbauche}}
\newcommand{\dateFinContrat}{\ensuremath\mathrm{dateFinContrat}}
\newcommand{\fonction}{\ensuremath\mathrm{fonction}}
\newcommand{\permanent}{\ensuremath\mathrm{permanent}}
\newcommand{\nonpermanent}{\ensuremath\mathrm{non-permanent}}
\newcommand{\emptystring}{\ensuremath\mathrm{vide}}
L'invariant de la classe \textsf{Employé} est le suivant:
\begin{tabbing}
M \= M \= M \= M \= M \= M \= M \kill
\> $\land$ \> $\identifiant \neq \nullvalue \land \identifiant \neq \emptystring$\\
\> $\land$ \> $\nom \neq \nullvalue \land \nom \neq \emptystring$\\
\> $\land$ \> $\prenom \neq \nullvalue \land \prenom \neq \emptystring$\\
\> $\land$ \> $\dateEmbauche \neq \nullvalue \land \fonction \neq \nullvalue$\\
\> $\land$ \> $\fonction = \permanent \implies \dateFinContrat = \nullvalue$\\
\> $\land$ \> $\fonction \neq \permanent \implies \dateFinContrat \neq \nullvalue$\\
\end{tabbing}

\newpage

\section{Préparation des tests unitaires}

{\color{red}\textbf{Les deux premières tables de décision sont à
    compléter (plus particulièrement les post-conditions).}}

\begin{table}[!ht]
\begin{center}
\begin{tabular}{|p{0.6\linewidth}|c|c|c|c|c|c|c|}
\hline
Numéro de test
&1&2&3&4&5&6&7\\
\hline
\hline
\texttt{identifiant} $\neq \nullvalue \land \neg \emptystring$
&F&T&T&T&T&T&T\\
\hline
\texttt{nom} $\neq \nullvalue \land \neg \emptystring$
& &F&T&T&T&T&T\\
\hline
\texttt{prénom} $\neq \nullvalue \land \neg \emptystring$
& & &F&T&T&T&T\\
\hline
\texttt{dateEmbauche} $\neq \nullvalue$
& & & &F&T&T&T\\
\hline
\texttt{fonction} $\neq \nullvalue$
& & & & &F&T&T\\
\hline
\texttt{fonction} $\in \{$direction département, direction adjointe
département, assistance gestion, enseignement recherche$\}$
& & & & & &F&T\\
\hline
\hline
$\identifiant' = \identifiant$
& & & & & & &T\\
\hline
$\nom' = \nom$
& & & & & & &T\\
\hline
$\prenom' = \prenom$
& & & & & & &T\\
\hline
$\dateEmbauche' = \dateEmbauche$
& & & & & & &T\\
\hline
$\dateFinContrat' = \nullvalue$
& & & & & & &T\\
\hline
$\fonction' = \fonction)$
& & & & & & &T\\
\hline
$\invariant$
& & & & & & &T\\
\hline
Levée d'une exception&\textsc{oui}&\textsc{oui}&\textsc{oui}&\textsc{oui}&\textsc{oui}&\textsc{oui}&\textsc{non}\\
\hline
\hline
Objet créé
&F&F&F&F&F&F&T\\
\hline
\hline
Nombre de jeux de test 
&2&2&2&1&1&4&4\\
\hline
\end{tabular}
\caption{Méthode \texttt{constructeurEmployé} de la classe
  \texttt{Employé} pour un employé permanent. Nous avons mis 4
  tests pour les jeux de test 6 et 7: 1 par fonction non autorisée et
  1 par fonction autorisée.}
\end{center}
\end{table}

\begin{table}[!ht]
\begin{center}
\begin{tabular}{|p{0.6\linewidth}|c|c|c|c|c|c|c|c|}
\hline
Numéro de test
&1&2&3&4&5&6&7&8\\
\hline
\hline
\texttt{identifiant} $\neq \nullvalue \land \neg \emptystring$
&F&T&T&T&T&T&T&T\\
\hline
\texttt{nom} $\neq \nullvalue \land \neg \emptystring$
& &F&T&T&T&T&T&T\\
\hline
\texttt{prénom} $\neq \nullvalue \land \neg \emptystring$
& & &F&T&T&T&T&T\\
\hline
\texttt{dateEmbauche} $\neq \nullvalue$
& & & &F&T&T&T&T\\
\hline
\texttt{dateFinContrat} $\neq \nullvalue$
& & & & &F&T&T&T\\
\hline
\texttt{dateFinContrat} $\geq$ \texttt{dateEmbauche}
& & & & &F&T&T&T\\
\hline
\texttt{fonction} $\neq \nullvalue$
& & & & & &F&T&T\\
\hline
\texttt{fonction} $\in \{$doctorat, post-doctorat, ingénierie
recherche, stage$\}$
& & & & & & &F&T\\
\hline
\hline
$\identifiant' = \identifiant$
& & & & & & & &T\\
\hline
$\nom' = \nom$
& & & & & & & &T\\
\hline
$\prenom' = \prenom$
& & & & & & & &T\\
\hline
$\dateEmbauche' = \dateEmbauche$
& & & & & & & &T\\
\hline
$\dateFinContrat' = \dateFinContrat$
& & & & & & & &T\\
\hline
$\fonction' = \fonction)$
& & & & & & & &T\\
\hline
$\invariant$
& & & & & & & &T\\
\hline
Levée d'une exception&\textsc{oui}&\textsc{oui}&\textsc{oui}&\textsc{oui}&\textsc{oui}&\textsc{oui}&\textsc{oui}&\textsc{non}\\
\hline
\hline
Objet créé
&F&F&F&F&F&F&F&T\\
\hline
\hline
Nombre de jeux de test 
&2&2&2&1&1&2&4&4\\
\hline
\end{tabular}
\caption{Méthode \texttt{constructeurEmployé} de la
classe \texttt{Employé} pour un employé non-permanent. Nous avons mis 2
  tests pour le jeu de test~6: 1 pour l'égalité de date et 1 lorsque
  les dates ne sont pas égales. Nous avons mis 4 tests pour les jeux
  de test 7 et 8: 1 par fonction non autorisée et 1 par fonction
  autorisée.}
\end{center}
\end{table}

\end{document}
